% ******************************
% General
% *****************************

% Include author names in TOC + Chapter Headings
\makeatletter

\let\chauth=\@empty
\renewcommand{\author}[1]{\def\chauth{#1}}

\titleformat{\chapter}[display]%
  {\relax}{\mbox{}\oldmarginpar{\vspace*{-3\baselineskip}\color{CTsemi}\chapterNumber\thechapter}}{0pt}%
  {\raggedright\spacedallcaps}[\normalsize\vspace*{.8\baselineskip}\titlerule%
                                \vspace*{-2.32\baselineskip}%
                                \begin{flushright}%
                                  {\itshape\chauth}%
                                \end{flushright}%
                              ]

\makeatother

% Drop chapter number from section headings, equations,
\renewcommand*\thesection{\arabic{section}}
\renewcommand*\theequation{\arabic{equation}}
\renewcommand*\thetable{\arabic{table}}
\renewcommand*\thefigure{\arabic{figure}}

% Convenience macro for adding chapters
\newcommand{\addchapter}[3]{%
\begin{refsection}%
  \author{#2}%
  \chapter{#1}%
  \cftaddtitleline{toc}{chapter}{\hspace{3em}\itshape#2}{}%
  \input{#3}%
  \markright{\spacedlowsmallcaps{References}}%
  \printbibliography[heading=subbibliography]%
\end{refsection}
}

\emergencystretch=1em

\DeclareCiteCommand{\citeyearhyper}
  {}
  {\bibhyperref{\printdate}}
  {\multicitedelim}
  {}
\newcommand{\citepos}[1]{\citeauthor{#1}'s (\citeyearhyper{#1})}

\renewcommand\labelitemi{$\circ$}
\renewcommand\labelitemii{$\diamond$}

\newcommand{\given}{\mathbin{|}}
%\usepackage{MnSymbol}

\newcommand*{\definitionautorefname}{Definition}

%***********
% Mahtani 
%***********

\usepackage{nicefrac}


%********
% Thoma 
%********

\usepackage{amssymb}
\usepackage{mathrsfs}
\usepackage{hyperref}
\usepackage{tikz}
\usepackage{pgf}
\usepackage{pgfplots}
\usepackage{amsmath}
\usepackage{graphicx}
\usepackage{pgfplots}
\usepackage{subcaption}
\usepackage{placeins}
\usepackage{tabularx}
\usepackage{csquotes}

\tikzstyle{decision} = [rectangle, minimum height=18pt, minimum width=18pt, draw=black, fill=none, thick, inner sep=0pt]
\tikzstyle{chance} = [circle, minimum width=18pt, draw=black, fill=none, thick, inner sep=0pt]
\tikzstyle{line} = [draw=none]

\tikzset{
grow=down,
sloped,
join=miter,
level 1/.style={sibling distance=3 cm,level distance=3.5 cm},
level 2/.style={sibling distance=3 cm, level distance=3.5 cm},
level 3/.style={sibling distance=3 cm, level distance=3.5 cm},
edge from parent/.style={thick, draw=black},
edge from parent path={(\tikzparentnode.south) -- (\tikzchildnode.north)},
every node/.style={text ragged, inner sep=1mm}
}

% \usetikzlibrary{automata,arrows,shapes,snakes,topaths,trees}
\usetikzlibrary{automata,arrows,shapes,decorations,topaths,trees}
%\usetikzlibrary{decorations.pathmorphing}

\pgfplotsset{compat=1.7} % added in response to backwards compatibility mode warning


%************
% Wenmackers 
%************

\usepackage{commath}


%************
% Easwaran 
%************

\usepackage{tikz-cd}
\usetikzlibrary{cd}
\newtheorem{theorem}{Theorem}
\newtheorem{definition}{Definition}


%********
% Konek 
%********

\usetikzlibrary{arrows, decorations.markings}
\usetikzlibrary{matrix}

\tikzstyle{vecArrow} = [thick, decoration={markings,mark=at position
   1 with {\arrow[semithick]{open triangle 60}}},
   double distance=1.4pt, shorten >= 5.5pt,
   preaction = {decorate},
   postaction = {draw,line width=1.4pt, white,shorten >= 4.5pt}]
\tikzstyle{innerWhite} = [semithick, white,line width=1.4pt, shorten >= 4.5pt]

\newcommand*\ov[1]{\overline{#1}}
\newcommand*\un[1]{\underline{#1}}
\newcommand*\tre[1]{\trianglerighteq{#1}}

\newcommand{\A}{\mathcal{A}}
\newcommand{\B}{\mathcal{B}}
\newcommand{\C}{\mathcal{C}}
\newcommand{\D}{\mathcal{D}}
\newcommand{\E}{\mathcal{E}}
\newcommand{\F}{\mathcal{F}}
\newcommand{\G}{\mathcal{G}}
\newcommand{\I}{\mathcal{I}}
\newcommand{\calP}{\mathcal{P}}
\newcommand{\Q}{\mathcal{Q}}
\newcommand{\R}{\mathcal{R}}
\newcommand{\T}{\mathcal{T}}
\newcommand{\U}{\mathcal{U}}
\newcommand{\V}{\mathcal{V}}
\newcommand{\X}{\mathcal{X}}
\newcommand{\Y}{\mathcal{Y}}

\newcommand{\Exp}{\mathrm{Exp}}
\newcommand{\est}{\textit{est}}

\newcommand{\fraka}{\mathfrak{a}}
\newcommand{\frakg}{\mathfrak{g}}


%******
% Lin 
%******

%\usepackage{epsfig, amsthm, amsmath, amssymb, latexsym, xpatch}
\usepackage[all]{xy}

%********
% Huber
%********

\usepackage{stmaryrd}
\usepackage{xcolor}
\usepackage[eulergreek]{sansmath}
\usepackage[euler]{textgreek}
\usetikzlibrary{calc}

\newcommand{\powerset}{\raisebox{.15\baselineskip}{\Large\ensuremath{\wp}}}
\newcommand{\powersetsubscr}{\raisebox{.15\baselineskip}{\small\ensuremath{\wp}}}
\newcommand{\ctrct}{\mathbin{\dot-}}
\def\dotminus{\mathbin{\ooalign{\hss\raise1ex\hbox{.}\hss\cr
  \mathsurround=0pt$-$}}}



%*********
% Genin
%*********

\usepackage{cancel}


%*********
% Briggs 
%*********

\usepackage{textcomp}
\usepackage{tikz-qtree}
\usepackage{rotating}
\DeclareSymbolFont{symbolsC}{U}{txsyc}{m}{n}
\DeclareMathSymbol{\boxright}{\mathrel}{symbolsC}{128}
