%*******************************************************
% Preface
%*******************************************************
\pdfbookmark[1]{Acknowledgments}{acknowledgments}


\begin{flushright}
    \textit{Overall, and ultimately, mathematical methods\\
    are necessary for philosophical progress\ldots\\}
    \smallskip
    {\small --- Hannes Leitgeb}

    \bigskip

    \textit{There is no mathematical substitute for philosophy.\\}
    \smallskip
    {\small --- Saul Kripke}
\end{flushright}

\bigskip

\begingroup
\let\clearpage\relax
\let\cleardoublepage\relax
\let\cleardoublepage\relax
\chapter*{Preface}

In formal epistemology, we use mathematical methods to explore the questions of epistemology and rational choice. What can we know? What should we believe and how strongly? How should we act based on our beliefs and values?

We begin by modelling phenomena like knowledge, belief, and desire using mathematical machinery, just as a biologist might model the fluctuations of a pair of competing populations, or a physicist might model the turbulence of a fluid passing through a small aperture. Then, we explore, discover, and justify the laws governing those phenomena, using the precision that mathematical machinery affords.

For example, we might represent a person by the strengths of their beliefs, and we might measure these using real numbers, which we call credences. Having done this, we might ask what the norms are that govern that person when we represent them in that way. How should those credences hang together? How should the credences change in response to evidence? And how should those credences guide the person’s actions? This is the approach of the first six chapters of this handbook.

In the second half, we consider different representations---the set of propositions a person believes; their ranking of propositions by their plausibility. And in each case we ask again what the norms are that govern a person so represented. Or, we might represent them as having both credences and full beliefs, and then ask how those two representations should interact with one another.

This handbook is incomplete, as such ventures often are. Formal epistemology is a much wider topic than we present here. One omission, for instance, is social epistemology, where we consider not only individual believers but also the epistemic aspects of their place in a social world. Michael Caie's entry on doxastic logic touches on one part of this topic, but there is much more. Relatedly, there is no entry on epistemic logic, nor any on knowledge more generally. There are still more gaps.

These omissions should not be taken as ideological choices. This material is missing, not because it is any less valuable or interesting, but because we failed to secure it in time. Rather than delay publication further, we chose to go ahead with what is already a substantial collection. We anticipate a further volume in the future that will cover more ground.

Why an open access handbook on this topic? A number of reasons. The topics covered here are large and complex and need the space allowed by the sort of 50 page treatment that many of the authors give. We also wanted to show that, using free and open software, one can overcome a major hurdle facing open access publishing, even on topics with complex typesetting needs. With the right software, one can produce attractive, clear publications at reasonably low cost. Indeed this handbook was created on a budget of exactly \textsterling$0$ ($\approx \$ 0$).

Our thanks to PhilPapers for serving as publisher, and to the authors: we are enormously grateful for the effort they put into their entries.

\bigskip

\hfill R. P. \,\&\, J. W.

\endgroup
